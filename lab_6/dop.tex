\documentclass{article}
\usepackage{blindtext}
\usepackage{multicol}
\usepackage[left=1.75cm,right=1.75cm,top=1cm,bottom=1cm]{geometry}
\usepackage{tikz}
\usepackage[utf8]{inputenc}
\usepackage{graphicx}
\usepackage{float}
\usepackage[russian]{babel}
\begin{document}
\begin{multicols}{2}

\begin{tikzpicture}
    \draw[red,dashed, ultra thick] (0,0)--(2,4);
    \draw[red,thick] (2,4)--(4,0);
    \node at (-0.5,0) {$D$};
    \node at (2,4.4) {$B$};
    \node at (2,-0.2) {$A$};
    \node at (4.5,0) {$C$};
    \draw[black,thick] (2,0)--(4,0);
    \draw[black,dashed, ultra thick] (0,0)--(2,0);
    \draw[blue,thick] (2,4)--(2,0);
    \begin{scope}
    \clip (-2,0) rectangle (2,1.5);
    \draw (2,0) circle(1);
    \fill[color=orange] (2,0) circle(1);

\end{scope}
\end{tikzpicture}



\begin{minipage}{0.5\textwidth}
  \begin{flushright}
\textbf{74 КНИГА I ПРЕДЛ.XLVIII.ТЕОРЕМА}
   
\begin{figure}[H]
\begin{subfigure}
        \includegraphics[width=0.1\textwidth]{so.png}
        сли в треугольнике квадрат одной стороны 
        \begin{tikzpicture}
            \draw[orange, ultra thick] (0,0)--(2,0);
            \node at (-0.2,0) {$B$};
            \node at (2.3,0) {$C$};
        \end{tikzpicture}
        равен сумме квадратов двух других сторон
        \begin{tikzpicture}
            \draw[blue, ultra thick] (0,0)--(2,0); 
            \node at (-0.2,0) {$A$};
            \node at (2.3,0) {$B$};
        \end{tikzpicture}
        \textbf{и}
        \begin{tikzpicture}
            \draw[black, ultra thick] (0,0)--(2,0); \node at (-0.2,0) {$A$};
            \node at (2.3,0) {$C$};
        \end{tikzpicture}
        ,то угол 
        заключенный между этими двумя сторонами прямой.
        \centering
            Проведем
            \begin{tikzpicture}
              \draw[black,dashed, ultra thick] (0,0)--(1,0);  
              \node at (-0.2,0) {$A$};
            \node at (1.1,0) {$D$};
            \end{tikzpicture}
            \begin{tikzpicture}
                \draw[black,thick] (0,0) -- (0,0.5);
                \draw[black,thick] (-0.2,0) -- (0.2,0);
            \end{tikzpicture}
            \begin{tikzpicture}
                \draw[blue, ultra thick] (0,0)--(1,0);
                \node at (-0.2,0) {$A$};
            \node at (1.1,0) {$B$};  
            \end{tikzpicture}
            
            и = 
            \begin{tikzpicture}
                \draw[black, ultra thick] (0,0)--(1,0);  
                \node at (-0.2,0) {$A$};
            \node at (1.1,0) {$C$};
            \end{tikzpicture}
            (пр. I.11, I.3)

            также проведем
            \begin{tikzpicture}
                \draw[orange,dashed, ultra thick]
                (0,0)--(1,0);  
                \node at (-0.2,0) {$B$};
            \node at (1.1,0) {$D$};
            \end{tikzpicture}
            
            Поскольку 
        \begin{tikzpicture}
                \draw[black,dashed, ultra thick] (0,0)--(1,0); 
                \node at (-0.2,0) {$A$};
            \node at (1.1,0) {$D$};
            \end{tikzpicture}
        =
        \begin{tikzpicture}
                \draw[black, ultra thick] (0,0)--(1,0);  \node at (-0.2,0) {$A$};
            \node at (1.1,0) {$C$};
            \end{tikzpicture}
        (постр.)

        \begin{tikzpicture}
                \draw[black,dashed, ultra thick] (0,0)--(1,0);  
                \node at (-0.2,0) {$A^2$};
            \node at (1.1,0) {$D^2$};
            \end{tikzpicture}
        =
        \begin{tikzpicture}
                \draw[black, ultra thick] (0,0)--(1,0);  
                \node at (-0.2,0) {$A^2$};
            \node at (1.1,0) {$C^2$};
            \end{tikzpicture}
        ;

        \therefore
        \begin{tikzpicture}
                \draw[black,dashed, ultra thick] (0,0)--(1,0);  
                \node at (-0.2,0) {$A$};
            \node at (1.1,0) {$D^2$};
            \end{tikzpicture}
        +\begin{tikzpicture}
                \draw[blue, ultra thick] (0,0)--(1,0);  \node at (-0.2,0) {$A$};
            \node at (1.1,0) {$B^2$};
            \end{tikzpicture}
        =\begin{tikzpicture}
                \draw[black, ultra thick] (0,0)--(1,0);  
                \node at (-0.2,0) {$A$};
            \node at (1.1,0) {$C^2$};
            \end{tikzpicture}
        +\begin{tikzpicture}
                \draw[blue, ultra thick] (0,0)--(1,0);  \node at (-0.2,0) {$A$};
            \node at (1.1,0) {$B^2$};
            \end{tikzpicture}

        но \begin{tikzpicture}
                \draw[black,dashed, ultra thick] (0,0)--(1,0);  
                \node at (-0.2,0) {$A$};
            \node at (1.1,0) {$D^2$};
            \end{tikzpicture}
        +\begin{tikzpicture}
                \draw[blue, ultra thick] (0,0)--(1,0);  \node at (-0.2,0) {$A$};
            \node at (1.1,0) {$B^2$};
            \end{tikzpicture}
        =\begin{tikzpicture}
                \draw[orange,dashed, ultra thick] (0,0)--(1,0); 
                \node at (-0.2,0) {$B$};
            \node at (1.1,0) {$D^2$};
            \end{tikzpicture}
        (пр. I.47),
        
        и \begin{tikzpicture}
                \draw[black, ultra thick] (0,0)--(1,0); 
                \node at (-0.2,0) {$A$};
            \node at (1.1,0) {$C^2$};
            \end{tikzpicture}
        +\begin{tikzpicture}
                \draw[blue, ultra thick] (0,0)--(1,0);  \node at (-0.2,0) {$A$};
            \node at (1.1,0) {$B^2$};
            \end{tikzpicture}
        =\begin{tikzpicture}
                \draw[orange, ultra thick] (0,0)--(1,0);  
                \node at (-0.2,0) {$B$};
            \node at (1.1,0) {$C^2$};
            \end{tikzpicture}
            (гип.)
            
        \therefore \begin{tikzpicture}
                \draw[orange,dashed, ultra thick] (0,0)--(1,0);  
                \node at (-0.2,0) {$B$};
            \node at (1.1,0) {$D^2$};
            \end{tikzpicture}
        =\begin{tikzpicture}
                \draw[orange, ultra thick] (0,0)--(1,0); 
                \node at (-0.2,0) {$B$};
            \node at (1.1,0) {$C^2$};
            \end{tikzpicture}

        \therefore \begin{tikzpicture}
                \draw[orange,dashed, ultra thick] (0,0)--(1,0); 
                \node at (-0.2,0) {$B$};
            \node at (1.1,0) {$D$};
            \end{tikzpicture}
        =\begin{tikzpicture}
              \draw[orange, ultra thick] (0,0)--(1,0);  \node at (-0.2,0) {$B$};
            \node at (1.1,0) {$C$};
            \end{tikzpicture}

        и \therefore \begin{tikzpicture}
    \begin{scope}
    \clip (0,0) rectangle (2,1.5);
    \draw (2,0) circle(1);
    \fill[color=orange] (2,0) circle(1);


\end{scope}
\node at (0.8,0) {$D$};
\node at (2,0) {$A$};
\node at (2,1.1) {$C$};
\end{tikzpicture}
= \begin{tikzpicture}
    \begin{scope}
    \clip (0,0) rectangle (2,1.5);
    \draw (2,0) circle(1);
    \fill[color=red] (2,0) circle(1);

\end{scope}
\end{tikzpicture}
(пр. I.8),

следовательно
\begin{tikzpicture}
    \begin{scope}
    \clip (1,-2) rectangle (-2,1);
    \draw (2,0) circle(1);
    \fill[color=red] (2,1.5) circle(1);

\end{scope}
\end{tikzpicture}
прямой угол.
ч. т. д.
        
        
\end{subfigure}
\end{figure} 
\end{flushright}
\end{minipage}
\end{multicols}


\end{document}
