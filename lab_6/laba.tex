\begin{multicols}{2}
\begin{figure}[t!]
    \begin{subfigure}[b]{0.385\textwidth}
        \centering
        \includegraphics[width=1\textwidth]{aaaaa.eps}
\captionsetup{labelformat=empty,singlelinecheck=false}
        \caption{{\textit{Рис. 18}}}
    \end{subfigure}
\end{figure}
    \noindent\textbf{Из уравнения теплового баланса} c_{1}m_{1}(t_{1}-t_{2})+ +m_{2}\lambda= c_{2}m_{2}(t_{0}-t_{2}) \textbf{ найдем массу } m_{x} \textbf{ замерзшей}
    
    \noindent\textbf{воды.}
        
    \noindent\textbf{Таким образом, объем содержимого калориметра будет равен}
    
    \begin{center}
    V = \tfrac{m_{1}-m_{x}}{Q_{1}}+\tfrac{m_{2}+m_{x}}{Q_{2}}=7,5*$10^-^3$
        \textbf{$m^3$.}   
    \end{center}
    \noindent\textbf{3. Для начального расположения справедливы соотношения}
    \begin{center} 
    1/d+1/f=1/F, d+f=a, f/d=Г.
    \end{center} 
    \noindent\textbf{В первом случае -}
    \begin{center} 
    1/d+1/f=1/F, d+f=a, f/d=Г.
    \end{center}
    \noindent\textbf{В первом случае -}
    
    \textbf{1/d_{1}+1/f_{1}=1/F, d_{1}+f_{1}=a+b, f_{1}/d_{1}=\textbf{Г}_{1};}
    
    \noindent\textbf{Отсюда находим}
    
     {Г}_{1}\approx9,1; \textbf{Г}_{2}=9.   
 
    
    \noindent\textbf{Знаковые конструкции}
    
    \noindent\textbf{("Квант" №9)}

    \noindent\textbf{1. Пусть S - сумма девяти искомых чисел , n - наименьшая из их попарных сумм. Подсчитав сумму всех попарных сумм двумя способами, получаем 8S=(2n+35) * 18, что невозможно.}

    \noindent\textbf{2. а) Так как сумма чисел каждой группы кратна трем, то и сумма всех чисел, равная} \tfrac{n(n+1)}{2} \textbf{, кратна трем. Значит, либо n, либо n+1}
    
    \noindent\textbf{кратно трем. А для таких n построить искомые разбиения нетрудно:}
    \begin{center}
    \textbf{n=3a: (1,2,3); (4,5,6); ...;(3a-2,3a-1,3a);}
    
    \textbf{n=3a-1: (1,2);(3,4,5); ...;}

    \textbf{(3a-3,3a-2,3a-1)}
    \end{center}
    \noindent\textbf{б) Так как сумма чисел каждой группы кратна четырем, то либо n, то либо n+1 кратно восьми. Для этих n укажем искомые разбиения. Восемь подряд идущих натуральных чисел легко разбить требуемым образом на две группы:}
    \begin{center}
        \textbf{(a+1,a+3,a+4,a+8);}
        
        \textbf{(a+2,a+5,a+6,a+7).}
    \end{center}
    \noindent\textbf{Если n кратно восьми , то искомое разбиение теперь очевидно. Если же n+1 кратно восьми, то для построения искомого примера достаточно заметить, что группа чисел от 1 до 7 обладает требуемым свойством: 3*7=1+2+...+6.}
    
    \noindent\textbf{3. Множество А состоит из нуля и всех натуральных чисел, в двоичной записи которых единица находится в разрядах, дающих остаток 1 при делении на 3(считая справа).Множество чисел В состоит из нуля и всех натуральных чисел, в двоичной записи которых единица находится в разрядах,дающих остаток 2 при делении на 3. Множество С состоит из нуля и всех натуральных чисел, в двоичной записи которых единица находится в разрядах кратных трем.}

    \noindent\textbf{4. Ясно, что либо n, либо n+1 кратно четырем. Если n=4a, то искомое разбиение таково: (1,4,5,8 ...,4а-3,4а);(2,3,6,7, ...,4a-2,4a-1).Если n-4a-1, то разбить на две разные по количеству гирь кучки невозможно. Разбиение же по массам таково:(1,2,4,7, ...,4a-4,4a-1);(3,5,6,...,4a-3,4a-2).}

    \noindent\textbf{5. Переход осуществляется от набора из n гирь к набору из (n+8) гирь. Дело в том, что имеет место равенство}
    \begin{center}
     
    \sqrt[{e^{-e}}]{(n+1)^2+(n+4)^2+(n+6)^2+(n+7)^2}=
    
    = {\iiint_{-\infty}^{+\infty}{(n+2)^2+(n+3)^2+(n+5)^2+(n+8)^2}
    \end{center}

    \noindent\textbf{6. Естественно было бы попытаться разбить требуемым образом набор из девяти гирь с весами} $n+1^2$,$n+2^2$,...,$n+9^2$ \textbf{. Но, к сожалению, это невозможно.Удается получить только "почти" требуемое разбиение:}
    \begin{center} 
    \begin{tabular}{|p{3cm}|p{1.3cm}|p{1.3cm}|p{1.3cm}|}
    \hline
    1 группа & $(n+1)^2$ & $(n+6)^2$ & $(n+8)^2$\\
    \hline
    2 группа & $(n+2)^2$ & $(n+4)^2$ & $(n+9)^2$\\  
    \hline
    3 группа & $(n+3)^2$ & $(n+5)^2$ & $(n+7)^2$\\
    \hline
    \end{tabular}
    \end{center}

    \noindent\textbf{При этом в первой и второй группах массы одинаковы, а в третьей - на 18 граммов меньше. Теперь становится понятно, как разбить требуемым образом на три группы набор из 27 гирь. Кучку из первых девяти гирь разобьем указанным образом, следующую кучку из девяти гирь разобьем так, чтобы легче была вторая группа, и последнюю кучку разложим так, чтобы легче была первая группа. Объединив затем все первые, все вторые и все третьи группы, получим требуемое разбиение набора из 27 гирь.}

    \textbf{ }
    
    \noindent\textbf{"Квант" для младших школьников}
    
    \noindent\textbf{("Квант" №9)}

    \noindent\textbf{1. Яблоки стоилк 45 копеек за килограмм.}

    \noindent\textbf{2. Пусть задачу решило х мальчиков, в девочек было k. Тогда девочек, решивших задачу было k-х, а всего решило задачу х+k-х=k человек. Значит, количество решивших задачу равно числу девочек.}

    \noindent\textbf{3. 47 486+7486+486+86+6=55 550.}

    \noindent\textbf{4. Вырежем нз бумаги параллелограмм, у которого одна сторона равна длине окружности основания первого цилиндра, вторая - окружности основания второго цилиндра, а высоты параллелограмма соответственно равны высотам цилиндров. Площадь такого параллелограмма равна 100 $cm^2$. Нетрудно видеть,что этни параллелограммом можно оклеить боковую поверхность как первого, так и второго цилиндра.}

    \noindent\textbf{5. Поскольку в сумме получались нечетные числа, человек шел по нечетной стороне улицы к каждый раз складывал нечетное число номе-}
    \begin{itemize}
  \item Это маркированный
  \item список
  \item :)
\end{itemize}
    
\end{multicols}
